\documentclass[]{IEEEtran}
\usepackage{url}
\usepackage{verbatim}

% Title Page
\title{Requirements Analysis}
\author{Jay Ng, Suzanne Candanedo, Adam Dodson, Heather Clark, Amber Rosevear, Ash Brent-Carpenter}

\begin{document}
	\maketitle
	
	\begin{comment}
		Each team is required to provide a Requirements Analysis Report. This will document your teams understanding of the specification, and will raise any questions that you may have. As well as addressing the specification itself, the report should also document any decisions that have been made in terms of group management (i.e. how you have organized yourselves to meet the goals of this project, how often you will meet, how you will address decision making etc.) and what methods you intend using in the planning and design of your solution.
		
		The requirements analysis report is not a design document, but rather a means by which you can demonstrate your understanding of the project’s aims. Any queries that you have regarding the specification should be clearly stated in this report.
		
		This part of your first deliverable must not exceed four sides of A4. Any appendices or other attachments will be removed.
	\end{comment}

	\section{Preface}
	% Details history of the document and who is expected to read it
	
	\section{Introduction}
	% Justifies the need for the system and outlines what it will do
	
	%\section{Glossary}
	
	\section{User Requirements}
	% Describe the services provided for the users
	% Written in natural language with diagrams

	\section{System Architecture}
	% Presents a high-level overview of the system, showing the distribution of functions across system modules
	
	\section{System Requirements Specification}
	% Describes functional and non-functional requirements
	% must, should and could
	% describe for developer and customer
	\subsection{Functional}
	\subsection{Non Functional}
	
	\section{System Models}
	% Shows relationships between the system components, usually through diagrams (object models, data-flow diagrams etc.)
	
	\section{System Evolution}
	% Describes assumptions on which the system is based and anticipated	changes due to changing user needs, hardware evolution
	
	\cite{test}

\bibliography{IEEEabrv,rabib}
\bibliographystyle{IEEEtran}
\end{document}          
