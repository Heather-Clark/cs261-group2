\documentclass[]{IEEEtran}
\usepackage{url}
\usepackage{verbatim}

% Title Page
\title{Planning and Design Document}
\author{Jay Ng, Suzanne Candanedo, Adam Dodson, Heather Clark, Amber Rosevear, Ash Brent-Carpenter}

\begin{document}
	\maketitle
	
	\begin{comment}
	You are also required to submit a Planning and Design Document. This should address the design of your software solution together with a detailed plan as to how (and when) this design is to be implemented. There are a number of things to consider in the design of software systems, including:
	
	Extensibility – how you can extend your solution given increasing demands; Robustness – how your solution will tolerate unpredictable or invalid input;
	
	Reliability – how the system will perform under everyday conditions;
	
	Correctness – how accurately your solution meets the requirements of the customer;
	
	Compatibility and portability – how easy your proposed system is to install and execute;
	
	Modularity and reuse – how well your system is divided into independent components and whether you have reused existing code;
	
	Security – whether your system can withstand hostile acts and influences;
	
	Fault-tolerance – whether your system can withstand and recover from component failure.
	
	This list is not exhaustive and there will be other concerns which your group will want to address. You will also want to adopt trusted design patterns or design methodologies to provide a template for the actual design of your system. No one method will be favoured (by the assessors) over another, but you will be graded on the process of selecting an appropriate methodology and your use of it.
	
	Each group must submit one Planning and Design Document and the report itself must not exceed ten sides of A4. Any appendices or other attachments will be removed.
	\end{comment}
	
	\begin{abstract}
	\end{abstract}

	\section{Introduction}
	
	\section{Implementation}
	
	\section{Design}

	\cite{test}
	
\bibliography{IEEEabrv,paddbib}
\bibliographystyle{IEEEtran}
\end{document}          
